\documentclass{article}
\usepackage{standalone}
\usepackage[catalan,english]{babel}
\usepackage{amsmath,amssymb,amsthm,mathtools}
\usepackage[sorting=none,maxnames=10]{biblatex}
\usepackage[left=2.5cm,right=2.5cm,top=3cm,bottom=3cm]{geometry}
\usepackage[colorlinks,linkcolor=blue,citecolor=blue,urlcolor=blue]{hyperref}
\usepackage{stmaryrd,csquotes}
\usepackage[affil-it]{authblk}
\usepackage{multirow}
\usepackage{physics}
\usepackage{enumitem}
\usepackage{epigraph}
\usepackage{indentfirst}
%%%%%%%%% more colors %%%%%%%%%%%%
\usepackage{xcolor}
\definecolor{darkblue}{RGB}{86, 40, 240}
\definecolor{lessgreen}{RGB}{0,100,0}
\definecolor{color1}{RGB}{255,158,1}
\definecolor{color2}{RGB}{255,74,1}
\definecolor{color3}{RGB}{220,0,0}
\definecolor{color4}{RGB}{180,0,0}
\usepackage[hypcap=false]{caption}
%%%%%%%%% tiks-picture %%%%%%%%%%%
\usepackage{tikz}
\usepackage{pgfplots}
\usepackage{mathtools}
\usetikzlibrary{patterns}
\pgfplotsset{compat = newest}
%%%%%%%%%%%%%%%%%%%%%%%%%%%%%%%%%%
\usepackage{subfig}

\newtheoremstyle{math}
    {\topsep}   % ABOVESPACE
    {\topsep}   % BELOWSPACE
    {}  % BODYFONT
    {}       % INDENT (empty value is the same as 0pt)
    {\bfseries} % HEADFONT
    {.}         % HEADPUNCT
    {5pt plus 1pt minus 1pt} % HEADSPACE
    {\thmname{#1}\thmnumber{ #2}\thmnote{ \bfseries(#3)}} % CUSTOM-HEAD-SPEC

\newtheoremstyle{TheoremNum}
    {\topsep}
    {\topsep}              %%% space between body and thm
    {}                      %%% Thm body font
    {}                              %%% Indent amount (empty = no indent)
    {\bfseries}                     %%% Thm head font
    {.}                             %%% Punctuation after thm head
    {5pt plus 1pt minus 1pt}                             %%% Space after thm head
    {\thmname{#1}\thmnote{ \bfseries #3}}%%% Thm head spec

\theoremstyle{math}
\newtheorem{definition}{Definició}[section]
\newtheorem{theorem}[definition]{Teorema}
\newtheorem{prop}[definition]{Proposició}
\newtheorem{lemma}[definition]{Lema}
\newtheorem{corollary}[definition]{Coro\lgem ari}

\theoremstyle{TheoremNum}
\newtheorem{prop*}[definition]{Proposició}
\newtheorem{corollary*}[definition]{Coro\lgem ari}

\newcommand\quot[2]{
    \mathchoice
        {% \displaystyle 
        \text{\raise1ex\hbox{$#1$}\!\Big/\!\lower1ex\hbox{$#2$}}}
        {% \textstyle
            #1/#2}
        {% \scriptstyle
            #1/#2}
        {% \scriptscriptstyle  
            #1/#2}
}% quotient group. Usage A/B--->\quot{A}{B}.

\newcommand{\0}{\ensuremath{\vb{0}}}
\newcommand{\N}{\ensuremath{\vb{N}}}
\newcommand{\X}{\ensuremath{\vb{X}}}
\newcommand{\Y}{\ensuremath{\vb{Y}}}
\newcommand{\Z}{\ensuremath{\vb{Z}}}
\newcommand{\NN}{\ensuremath{\mathbb{N}}} % set of real numbers
\newcommand{\RR}{\ensuremath{\mathbb{R}}} % set of real numbers
\newcommand\Hz{\text{ Hz}}
\DeclareMathOperator{\arccosh}{arccosh}

\addbibresource{references.bib}

%%% url symbol for references it is needed \usepackage{stmaryrd} %%%
\newcommand\enllas{\raise.5pt\hbox{$\boxempty\kern-4.85pt{}^{\nearrow}$}\kern-2pt}

\DeclareFieldFormat{url}{%
  \ifhyperref
    {\href{#1}{\enllas}}
    {\url{#1}}}
%%%%%%%%%%%%%%%%%%%%%%%%%%%%%%%%%%%%%%%%%%%%%%%%%%%%%%%%%%%%%%%%%%%%

%\setlength{\parindent}{0pt} % Indentation disabled

\title{\bfseries\large MESURES DE DISSONÀNCIA}

\author{Víctor Ballester Ribó, NIU:1570866\endgraf Oriol Bosquet Gallardo, NIU: 1571598\endgraf Carlo Sala Gancho, NIU: 1570775}
\date{\parbox{\linewidth}{\centering
  Taller de modelització\endgraf
  Grau en Matemàtiques\endgraf
  Universitat Autònoma de Barcelona\endgraf
  Juny de 2021}}
\begin{document}
\maketitle
\selectlanguage{english}
\begin{abstract}
  \noindent És ben sabut que hi ha combinacions de notes musicals que sonen millor que d'altres. Des de l'antiguitat se sap que això correspon al fet que certs paràmetres identificatius de les notes són proporcionals a nombres enters senzills. En aquest treball descriurem una modelització matemàtica de com mesurar el grau de dissonància produït quan toquem dos (o més) notes musicals simultàniament. Finalment, compararem els resultats amb l'opinió pública per veure si hi ha algun tipus de correspondència entre els resultats del model, els fets empírics i els fets teoritzats en l'antiguitat.
\end{abstract}
\thispagestyle{empty}
\newpage
\thispagestyle{empty}
\vspace*{\fill}
\vspace{-2cm}
\epigraph{It occurred to me by intuition, and music was the driving force behind that intuition. My discovery was the result of musical perception.}{\textit{Albert Einstein}}
\vspace*{\fill}
\newpage
\selectlanguage{catalan}
\setcounter{page}{1}
\tableofcontents
\newpage
\section{Introducció}
L'objectiu principal del treball és donar a entendre matemàticament per què es produeix la dissonància i intentar modelitzar-la quantitativament, per tal de determinar d'entre dues combinacions de notes musicals quina és la més dissonant. Per això començarem fent una petita introducció de teoria musical on aprofitarem per definir els conceptes necessaris i bàsics per al treball. Seguidament, comentarem molt breument certs conceptes referents a l'oïda humana que necessitarem per deduir una funció de dissonància. Un cop fet això, descriurem el model de dissonància que proposem juntament amb una anàlisi dels resultats obtinguts. Finalment, compararem aquests valors amb els recollits en un test fet prèviament a un públic general.
\section{Teoria musical i definicions prèvies}\label{teoria_musical}
Aquest treball està estretament relacionat amb la música, amb com els humans percebem els sons i, per extensió, amb les notes musicals. És per això que ens cal fer una petita incursió en la teoria musical per poder entendre correctament el sentit del nostre model.\par
És ben conegut per tothom que hi ha combinacions de notes musicals que sonen \textit{bé} i d'altres que sonen \textit{malament}. Ara bé, què vol dir \textit{bé} i \textit{malament} en aquest context? Vegem ara dues definicions que ens ho aclariran:
\begin{definition}
  Anomenem \textit{consonància} la qualitat de dos o més sons amb una relació de freqüències concreta, que sonen agradables a l'oïda humana.\par
  \noindent Anomenem \textit{dissonància} la qualitat de dos o més sons amb una relació de freqüències concreta, que sonen poc agradables a l'oïda humana.
\end{definition}
Quedant-nos amb aquestes definicions, descriurem en la secció \ref{teoria_auditiva} del treball quin és el motiu pel qual hi ha combinacions de sons que no són agradables a l'oïda humana, i en la secció \ref{model} intentarem donar una quantificació d'aquesta inharmonia.\par
Distingirem dos tipus de sons: els sons simples i els sons complexos. Les definicions d'aquests ens seran útils per caracteritzar-los:
\begin{definition}[So simple]
  Siguin $f\in(0,\infty)$ i $a\in[0,\infty)$. Anomenem \textit{so simple} o \textit{so pur} el parell $s=(f, a)$ on $s$ és el so sinusoidal d'equació: $$y_s(t)=a\sin(2\pi f)$$
\end{definition}
\noindent Observem que en aquesta definició hem omès una possible fase del so. Això ho hem suposat perquè quan toquem una nota musical (per exemple, una tecla d'un piano) tots els harmònics correponents a aquesta nota comencen a vibrar alhora, de manera que el desfasament entre cada parell d'harmònics és nul.\par D'altra banda, és ben sabut que l'amplitud té unitats de longitud. No obstant això, nosaltres la considerarem adimensional per facilitar-ne la manipulació. Més formalment, el que considerem no són amplituds, sinó relacions d'amplituds, que són magnituds adimensionals. És a dir, fixada una amplitud base $a_0$ i donat un so pur $s=(f,a)$ d'amplitud real $a^*$ prenem $a$ que sigui $a:=\frac{a^*}{a_0}$. Observem que d'aquesta manera, el so \textit{real} i el so \textit{modificat} satisfan la mateixa equació, llevat d'un escalar.
\begin{definition}
  Definim el conjunt $\mathcal{S}$ de sons simples com: $$\mathcal{S}=\{s=(f,a):f\in(0,\infty),a\in[0,\infty)\text{ i $s$ és el so d'equació }y_s(t)=a\sin(2\pi ft)\}$$
\end{definition}
\begin{definition}[So complex]\label{so_complex}
  Siguin $s_1,\ldots,s_n\in\mathcal{S}$ són simples tals que $s_i=(f_i,a_i)$ per $i=1,\ldots,n$. Anomenem \textit{so complex} el conjunt $\X=\{s_1,\ldots,s_n\}$ tal que $\X$ és el so d'equació: $$y_{\X}(t)=\sum_{i=1}^ny_{s_i}(t)=\sum_{i=1}^na_i\sin(2\pi f_it)$$
\end{definition}
\begin{definition}
  Definim el conjunt $\mathcal{C}$ de sons complexos com:
  $$\mathcal{C}=\left\{\X=\{s_1,\ldots,s_n\}:s_i\in\mathcal{S}\text{ per } i=1,\ldots,n\text{ i $\X$ és el so d'equació }y_{\X}(t)=\sum_{i=1}^ny_{s_i}(t)\right\}$$
\end{definition}
\noindent És a dir, el conjunt $\mathcal{C}$ està format per un conjunt de tuples de 2 elements (corresponents a sons simples). D'altra banda, tot i que hem definit els elements del conjunt $\mathcal{C}$ com sumes finites de sons simples, és natural pensar que els elements de $\mathcal{C}$ podrien tenir també cardinal infinit. Efectivament, això és possible, però no ho hem considerat a causa del poc interès pràctic que té aquest fet en relació amb l'objectiu principal del treball\footnote{L'oïda humana només és capaç de percebre sons les freqüències dels quals estan en l'interval entre $20\Hz$ i $20000\Hz$. És per això que a la pràctica considerar infinits harmònics és poc coherent.}.\par
Tot i que la definició \ref{so_complex} de so complex l'hem feta per a combinacions arbitràries de sons purs, podem  particularitzar-la amb l'objectiu d'apropar-nos més als propòsits d'aquest treball: les \textit{notes musicals}.
\begin{definition}
  Sigui $\N\in\mathcal{C}$ un so complex tal que $$\N=\{(f_1,a_1),\ldots,(f_n,a_n)\},$$ on $n\in\NN\cup\{\infty\}$. Diem que $\N$ és una \textit{nota musical} si es compleix que $f_k=kf_1$ per a tot $k=1,\ldots,n$. En aquest cas $n$ s'anomena \textit{nombre d'harmònics de $\N$} i el so simple $(f_k,a_k)$ també s'anomena \textit{harmònic $k$-èssim de $\N$}, per a $k=1,\ldots,n$. En particular, el so $(f_1,a_1)$ s'anomena \textit{harmònic fonamental de $\N$} i la seva freqüència, $f_1$, \textit{freqüència fonamental de $\N$}.
\end{definition}
\noindent No hem de confondre la definició \textit{contínua} que hem fet de nota musical amb la versió \textit{discreta} d'aquesta, per exemple, les notes en les tecles d'un piano. En el primer cas la freqüència fonamental de la nota pot variar lliurement en l'interval $(0,\infty)$, mentre que en el segon cas, hi ha un nombre finit de notes definides cadascuna amb una certa freqüència. Per no confondre els termes, aquest segon conjunt de notes, l'anomenarem \textit{notes de l'escala musical}.
\begin{definition}
  Definim el conjunt $\mathcal{N}$ de notes musicals com: $$\mathcal{N}=\left\{\X\in\mathcal{C}:\X\text{ és una nota musical}\right\}$$
\end{definition}
\noindent Si observem l'escala musical, veiem que hi ha únicament 12 noms de notes diferents\footnotemark\space que es van repetint seguint el mateix patró en tota l'escala. Cal observar també:
\footnotetext{Ens centrarem, d'ara endavant, en l'escala musical occidental, composta per les notes DO, DO\#, RE, RE\#, MI, FA, FA\#, SOL, SOL\#, LA, LA\#, SI.}
\begin{prop}
  Siguin $\N_1$, $\N_2$ dues notes de l'escala musical que tenen el mateix nom de nota i siguin $f_1$, $f_2$ les seves respectives freqüències fonamentals. Sense pèrdua de generalitat, suposem que $f_1 \leq f_2$. Llavors, $\frac{f_{2}}{f_{1}} = 2^n$ per a algun $n \in \mathbb{N}\cup\{0\}$.
\end{prop}
\begin{proof}
  Per fer que això passi, es declara que la relació entre les freqüències de dues notes successives sempre sigui de $\sqrt[12]{2}$\footnotemark. D'aquesta manera, com que hem dit que hi ha 12 notes diferents, quan tornem a arribar a una nota amb el mateix nom que l'original, haurem fet 12 salts de nota, és a dir, $\left(\sqrt[12]{2}\right)^{12} = 2$. Si féssim aquests 12 salts $n$ vegades, tindríem ${\left(\sqrt[12]{2}\right)}^{12\cdot n} = 2^n$. D'aquesta manera fem que es compleixi la proposició anterior. \footnotetext{Aquesta manera de definir la distància entre notes a fi que totes les notes siguin equidistants entre si s'anomena Temperament igual \cite{wikitemp}.}
\end{proof}
\section{Teoria auditiva}\label{teoria_auditiva}
El nostre cos ha desenvolupat un sistema complex per al reconeixement de sons, fruit de milers d'anys d'evolució. Això, juntament amb l'efecte de la cultura en la nostra vida, ha provocat que un cert conjunt de freqüències sigui més agradable per a la nostra oïda que un altre. Per tal de modelar en detall com percebem la dissonància, ens caldrà fer menció de dos conceptes clau en el sistema auditiu humà: la còclea i la membrana basilar. \par
La còclea és una estructura en forma de tub enrotllat en espiral, similar a la closca d'un cargol. Al centre i interior del tub es troba una membrana anomenada membrana basilar (vegeu figura \ref{coclea}). Aquesta membrana vibra quan ones de so arriben a ella. Sobre la membrana basilar es troba l'òrgan de Corti, que és el responsable de transformar les vibracions del so en impulsos nerviosos que són enviats al cervell.\par
Un concepte que serà important per a la caracterització del nostre model és el de \textit{banda crítica}. Aquesta és l'amplada de freqüències dins de la qual un segon so interferirà en la percepció d'un primer so, entenent que activaran les mateixes cè\lgem ules de l'òrgan de Corti. Dit d'una altra manera, quan una freqüència arriba a la membrana basilar, aquesta fa que s'activin els receptors corresponents a aquesta freqüència i els receptors en un entorn (la banda crítica) d'aquesta. Per tant, si dues freqüències arriben a la membrana basilar i no són prou distants entre si, es produeix una interferència que fa que no siguem capaços de distingir cada freqüència per separat i, com a conseqüència, es produeix dissonància. En canvi, si les dues freqüències estan prou allunyades entre si, el nostre cervell les interpreta com a diferents i provoca una resposta agradable: es produeix consonància (vegeu figura \ref{membrana}).\par
\begin{figure}[ht]
  \begin{minipage}[c]{0.49\linewidth}
    \centering
    \includegraphics[height=0.6\linewidth]{Imatges_beamer2/coclea.png}
    \caption{Estructura de la còclea \href{https://www.pinterest.com/pin/336995984614355654/}{\enllas}}
    \label{coclea}
  \end{minipage}
  \hfill
  \begin{minipage}[c]{0.49\linewidth}
    \centering
    \includegraphics[height=0.6\linewidth]{Imatges_beamer2/basilar_membrane.jpg}
    \caption{Interferència de dos sons en la membrana basilar \href{https://www.phys.uconn.edu/~gibson/Notes/Section7_3/Sec7_3.htm}{\enllas}}
    \label{membrana}
  \end{minipage}
\end{figure}
Això ens porta a concloure que percebem dues freqüències com a dissonants si es troben dins la mateixa banda i, com a consonants si es troben en bandes diferents. Per tant, és natural preguntar-se quina amplada té aquesta banda crítica, per tal d'estudiar amb més precisió quan dos sons sonen desagradablement.\par
Hi ha diversos models que parametritzen l'amplada de banda crítica corresponent a cada freqüència de l'espectre audible. El més adient per al nostre model és el que donen William Hutchinson i Leon Knopoff en \cite{hutchinson} basat en dades de \cite{plomp,goodwin,mayer} que parametritza l'amplada de banda crítica amb la funció:
\begin{equation}
  \text{CBW}(f)=1.72 f^{0.65}
  \label{CBW}
\end{equation}
Cal mencionar que perquè aquest fórmula tingui sentit (pel que fa a la dimensionalitat), hem de considerar la freqüència $f$ adimensional. Més formalment, en comptes de considerar $f^{0.65}$, es pot considerar una freqüència \textit{normalitzada} per a un valor fixat $f_0=1\Hz$, és a dir, considerar ${\left(\frac{f}{f_0}\right)}^{0.65}$. D'altra banda, la funció $\text{CBW}(f)$ retorna una amplada mesurada en Hz i, per tant, deduïm que la constant $k=1.72$ té unitats de Hz. D'aquesta manera resolem el problema de la dimensió de $\text{CBW}(f)$.
\section{Model per a la mesura de dissonància}\label{model}
\noindent Les demostracions de les proposicions que apareguin en aquesta secció es troben a l'annex \ref{demos}.
\subsection{Operacions definides a \texorpdfstring{$\mathcal{C}$}{C}}
Primer de tot, l'objectiu és definir operacions que ens permetin combinar dos (o més) sons complexos en un de sol i modificar la intensitat d'un so complex.
\begin{definition}
  Siguin $\X,\Y\in\mathcal{C}$. Definim l'operació \textit{suma $\oplus$} entre sons complexos com l'aplicació:
  \begin{equation*}
    \begin{array}{r@{\hspace{0.5\tabcolsep}}c@{\hspace{0.5\tabcolsep}}c@{\hspace{0.5\tabcolsep}}l}
      \oplus: & \mathcal{C}\times\mathcal{C} & \longrightarrow & \mathcal{C}           \\
              & (\X,\Y)                      & \longmapsto     & \X\oplus \Y:=\X\cup\Y
    \end{array}
  \end{equation*}
\end{definition}
\begin{definition}\label{prod_per_escalar}
  Sigui $\lambda\in[0,\infty)$ i $\X=\{(f_1,a_1),(f_2,a_2),\ldots,(f_n,a_n)\}\in\mathcal{C}$. Definim l'operació \textit{producte per escalar $\cdot$} entre un so complex i un escalar com l'aplicació:
  \begin{equation*}
    \begin{array}{r@{\hspace{0.5\tabcolsep}}c@{\hspace{0.5\tabcolsep}}c@{\hspace{0.5\tabcolsep}}l}
      \cdot: & \mathcal{C}\times[0,\infty) & \longrightarrow & \mathcal{C}         \\
             & (\X,\lambda)                & \longmapsto     & \lambda\cdot \X:=\Y
    \end{array}
  \end{equation*}
  on $\Y=\{(f_1,\lambda a_1),(f_2,\lambda a_2),\ldots,(f_n,\lambda a_n)\}\in\mathcal{C}$.
\end{definition}
És clar que les operacions $\oplus$ i $\cdot$ sobre el conjunt $\mathcal{C}$ estan ben definides en el sentit que donats dos sons $\X,\Y\in\mathcal{C}$, aleshores $\X\oplus\Y\in\mathcal{C}$ i $\lambda\cdot\X\in\mathcal{C}$. Efectivament, en el primer cas $\X\oplus\Y$ és, per definició, la unió de dos conjunts formats per sons simples, per tant, $\X\oplus\Y$ també serà un conjunt format per sons simples. En el segon cas ja ho hem vist en la pròpia definició.\par
D'altra banda, tot i que aquestes definicions les hem fetes entre sons complexos, com que $\mathcal{S}\subset\mathcal{C}$, els sons simples els podem pensar com a sons complexos. Per tant, té sentit també considerar sumes de sons simples i productes de sons simples per escalars, com apareixeran més endavant.
\subsection{Funció de dissonància per a sons simples}
Amb tot això ja podem definir una funció per mesurar el grau de dissonància entre dos sons complexos. Suposem, abans que res, que tenim una funció $\delta$ que mesura el grau de dissonància entre dos sons simples. És a dir, $\delta$ és una funció de la forma:
\begin{align*}
  \delta:\mathcal{S}\times\mathcal{S} & \longrightarrow\RR         \\
  (s_1,s_2)                           & \longmapsto\delta(s_1,s_2)
\end{align*}
Aquesta funció ha de complir necessàriament les propietats següents:
\begin{enumerate}[label=$\delta$\arabic*),ref=$\delta$\arabic*]
  \item\label{delta1} $\delta(s_1,s_2)=\delta(s_2,s_1)$ per a tot $s_1,s_2\in\mathcal{S}$.
  \item\label{delta2} $\delta(\lambda\cdot s_1,s_2)=\lambda\delta(s_1,s_2)$ per a tot $s_1,s_2\in\mathcal{S}$ i tot $\lambda\in[0,\infty)$.
  \item\label{delta3} $\delta(s_1,\lambda\cdot s_2)=\lambda\delta(s_1,s_2)$ per a tot $s_1,s_2\in\mathcal{S}$ i tot $\lambda\in[0,\infty)$.\par
\end{enumerate}
En aquestes propietats entenem que, anàlogament a la definició \ref{prod_per_escalar}, si $s_1=(f_1,a_1)$ i $s_2=(f_2,a_2)$, aleshores $\lambda\cdot s_1=(f_1,\lambda a_1)$ i $\lambda\cdot s_2=(f_2,\lambda a_2)$. Això, juntament amb el fet que l'amplitud de qualsevol so és sempre no-negativa, ens fa veure perquè $\lambda$ i $\mu$ no poden ser valors negatius. Observem, a més, que la tercera propietat és conseqüència de la primera i de la segona. D'altra banda, ajuntant les dues últimes propietats obtenim que $\forall\lambda,\mu\in[0,\infty)$ es compleix $\delta(\lambda\cdot s_1,\mu\cdot s_2)=\lambda\mu\delta(s_1,s_2)$, que és equivalent a dir: $$\delta(s_1,s_2)\propto a_1a_2$$
\subsection{Model de dissonància per a sons complexos}
\noindent L'objectiu ara és crear una funció
\begin{align*}
  D:\mathcal{C} & \longrightarrow\RR \\
  \X            & \longmapsto D(\X)
\end{align*}
que mesuri la dissonància d'un so complex. Per això definirem primer una altra funció $d$ que mesura la dissonància \textit{relativa} entre dos sons.
\begin{definition}
  Definim la \textit{mesura de la dissonància relativa $d$} entre dos sons complexos com:
  \begin{equation*}
    \begin{array}{r@{\hspace{0.5\tabcolsep}}c@{\hspace{0.5\tabcolsep}}c@{\hspace{0.5\tabcolsep}}l}
      d: & \mathcal{C}\times\mathcal{C}      & \longrightarrow & \RR                                                \\
         & (\{s_i\}_{i=1}^n,\{r_i\}_{i=1}^m) & \longmapsto     & \frac{1}{2}\sum_{i=1}^n\sum_{j=1}^m\delta(s_i,r_j)
    \end{array}
  \end{equation*}
\end{definition}
\noindent Aquesta funció té les propietats següents:
\begin{prop}\label{prop_dem1}
  Siguin $\X,\Y,\Z\in\mathcal{C}$ sons complexos i $\lambda\in[0,\infty)$. Llavors:
  \begin{enumerate}[label=$d$\arabic*),ref=$d$\arabic*]
    \item\label{d1} $d(\X,\Y)=d(\Y,\X)$.
    \item\label{d2} $d(\X\oplus\Y,\Z)=d(\X,\Z)+d(\Y,\Z)$.
    \item\label{d3} $d(\X,\Y\oplus\Z)=d(\X,\Y)+d(\X,\Z)$.
    \item\label{d4} $d(\lambda\cdot\X,\Y)=\lambda d(\X,\Y)$.
    \item\label{d5} $d(\X,\lambda\cdot\Y)=\lambda d(\X,\Y)$.
  \end{enumerate}
\end{prop}
\noindent Amb aquesta definició, donat un so complex $\X\in\mathcal{C}$, podem preguntar-nos què ocorre quan avaluem $d(\X,\X)$.
\begin{definition}[Mesura de dissonància]
  Definim la \textit{mesura de la dissonància $D$ d'un so complex} com:
  \begin{align*}
    D:\mathcal{C} & \longrightarrow\RR   \\
    \X            & \longmapsto d(\X,\X)
  \end{align*}
\end{definition}
\noindent Amb tals definicions obtenim una sèrie de propietats que seran importants a l'hora de calcular la dissonància entre dos (o més) sons de manera computacional\footnote{Per a més informació sobre el codi informàtic en llenguatge C vegeu \url{https://github.com/carlosala/dissonance}.}.
\begin{prop}\label{prop_dem2}
  Siguin $\X,\Y\in\mathcal{C}$ sons complexos i $\lambda\in[0,\infty)$. Llavors:
  \begin{enumerate}[label=$D$\arabic*),ref=$D$\arabic*]
    \item\label{D1} $D(\X\oplus\Y)=D(\X)+D(\Y)+2d(\X,\Y)$.
    \item\label{D2} $D(\lambda\cdot \X)=\lambda^2D(\X)$.
  \end{enumerate}
\end{prop}
\begin{corollary}\label{coro_dem3}
  Siguin $\X_1,\ldots,\X_n\in\mathcal{C}$ sons complexos i $\lambda\in[0,\infty)$. Llavors:
  \begin{equation}\label{eq_final}
    D(\X_1\oplus\cdots\oplus\X_n)=\sum_{i=1}^nD(\X_i)+\sum_{\substack{i,j=1\\i\ne j}}^nd(\X_i,\X_j)=\sum_{i,j=1}^nd(\X_i,\X_j)
  \end{equation}
\end{corollary}
\noindent Observem que la propietat \ref{D1} ens permet expressar el valor $d(\X,\Y)$ en funció únicament de $D$, de la següent manera: $$d(\X,\Y)=\frac{1}{2}\left[D(\X+\Y)-D(\X)-D(\Y)\right]$$
Així doncs, aquesta relació juntament amb la de la definició de la funció $D$ ($D(\X):=d(\X,\X)$) ens permet passar d'una funció a l'altra i viceversa.\par
Aquesta estreta relació entre $d$ i $D$, i el fet que ambdues tenen un domini de definició ``similar''\footnote{En aquest cas, el terme ``similar'' fa referència al fet que el domini d'una funció està format pel conjunt $\mathcal{C}$ i el domini de l'altra, pel producte cartesià d'aquest conjunt amb si mateix, és a dir, pel conjunt $\mathcal{C}\times\mathcal{C}$.} ens fa plantejar la possibilitat de ser $d$ una forma bilineal i $D$ la seva forma quadràtica associada. Perquè això fos possible s'hauria de complir que $(\mathcal{C},\oplus,\cdot)$ fos un $\RR$-espai vectorial, que no és cert\footnote{Una manera fàcil de veure que $(\mathcal{C},\oplus,\cdot)$ no és un $\RR$-espai vectorial és tenir en compte que donada una equació de la forma $y(t)=\sum_{i=1}^na_i\sin(2\pi f_it)$ hi ha més d'un element a $\mathcal{C}$ que la satisfà. En efecte, només cal considerar els sons complexos $\X=\emptyset$ i $\Y=(f,0)$, que són clarament diferents, i satisfan la mateixa equació $y_{\X}(t)=y_{\Y}(t)=0$. Això provoca que, entre d'altres coses, l'invers d'un element per l'operació $\oplus$ no sigui únic.\newline Si es volgués insistir a fer un espai vectorial a partir del conjunt $\mathcal{C}$ de sons complexos, caldria primer de tot modificar el conjunt de sons simples afegint-hi la fase com a propietat, juntament amb la freqüència i amplitud. D'altra banda, també s'hauria de definir una relació $\sim$ de forma que si $\X$ i $\Y$ fossin dos sons complexos amb les suposades condicions, aleshores $\X\sim \Y\iff y_{\X}(t)=y_{\Y}(t)$. És fàcil veure que aquesta relació seria d'equivalència. D'aquesta manera, es pot veure que el conjunt $\mathfrak{C}$ de les classes d'equivalència juntament amb les operacions $\oplus$ i $\cdot$ retocades convenientment per definir-les sobre $\mathfrak{C}$ (i no sobre $\mathcal{C}$) té estructura de $\RR$-espai vectorial.}. Per tant, $d$ i $D$ són simplement dues aplicacions que satisfan propietats de bilinealitat.
\subsection{Implementació de la funció \texorpdfstring{$\delta$}{delta}}
Fet aquest plantejament del model de la dissonància, podem observar que el valor final depèn únicament i exclusivament de la funció $\delta$, cosa que intensifica la importància de la seva tria.\par
Per fer-ho, adoptarem un punt de vista sensitiu: la banda crítica. A més, ens basarem en dades experimentals fetes per Plomp i Levelt \cite{plomp}, cosa que donarà una major credibilitat a la nostra fórmula.\par Com hem explicat a la secció \ref{teoria_auditiva}, quan dues freqüències sonen al mateix instant i s'activen els mateixos nervis de la membrana basilar, diem que les dues freqüències estan en la mateixa banda crítica i, per tant, es produeix interferència, que es tradueix en dissonància. Si les dues freqüències estan suficientment separades en el sentit que s'activen nervis diferents de la membrana basilar, les percebrem com a diferents i, per tant, no hi haurà a penes dissonància entre elles.\par
Com hem mencionat, per calcular una funció $\delta$ adequada ens hem basat en els estudis empírics de Plomp i Levelt \cite{plomp}. Plomp i Levelt van modelitzar empíricament la dissonància entre dos sons purs.
\begin{figure}[ht]
  \centering
  \includegraphics[width=0.4\linewidth, angle=-0.1]{Imatges_main/plompt-levelt.png}
  \caption{Resultats empírics de Plomp i Levelt \cite{plomp}}
  \label{fig_plomp}
\end{figure}\par
\noindent Algunes equacions funcionals que aproximen bé els seus resultats són les següents: $$\delta_1(x)=ae^{-\alpha x}-be^{-\beta x}\qquad\delta_2(x)=\alpha e^{-\left(\log(\beta x)\right)^2}\qquad\delta_3(x)=\beta xe^{-\beta x}$$
Aquestes funcions tenen en comú que, per a valors adequats dels paràmetres, creixen des de 0 fins a un valor màxim i a partir d'aquí decreixen de nou fins a zero. Per al nostre model finalment hem decidit adoptar-ne la tercera ja que l'hem vista més apropiada que la segona, i la primera ja ha estat utilitzada per William A. Sethares (vegeu \cite{sethares1})\footnote{Per a altres modelitzacions de corbes de dissonància, consulteu la feta per David J. Benson \cite{benson} usant també funcions exponencials, i la feta per Giorgio Dillon \cite{dillon} usant una funció polinòmica.}. Notem que l'elecció de la funció usada no és un fet objectiu, a causa de la falta de precisió de les dades de Plomp i Levelt i la subjectivitat de la percepció de la dissonància. L'objectiu principal que ha de satisfer la funció, més enllà de les propietats \ref{delta1}-\ref{delta3}, és que tingui unes característiques que s'ajustin bé als resultats empírics \cite{benson}.\par
Així doncs, donats dos sons simples $s_1=(f_1,a_1)$ i $s_2=(f_2,a_2)$, volem calcular una funció de dissonància $\delta(s_1,s_2)$. Com hem mencionat, per crear aquesta funció tindrem en compte el concepte de banda crítica, introduïda a la secció \ref{teoria_auditiva}. Segons dades de E. Zwicker, G. Flottorp i S. S. Stevens \cite{zwicker}, la màxima dissonància entre els sons $s_1$ i $s_2$ es troba quan les seves respectives freqüències ($f_1$ i $f_2$) estan separades aproximadament un 25\% de l'amplada de la banda crítica de la freqüència mitjana $f_m:=\frac{f_1+f_2}{2}$ (vegeu figura \ref{fig_plomp}). És a dir, les freqüències $f_1$ i $f_2$ tindran una dissonància màxima quan $\frac{|f_1-f_2|}{\text{CBW}(f_m)}=0.25$. Així doncs, per determinar el valor de $\beta$ (que dependrà de $f_1$ i $f_2$) de la funció $\beta xe^{-\beta x}$, hem de tenir en compte que s'ha d'assolir el màxim en la freqüència esmentada. Tenim doncs que:
$$\left(\beta xe^{-\beta x}\right)'=0\iff \beta e^{-\beta x}(1-\beta x)=0\iff\beta x=1$$ A més, com que $$\left(\beta xe^{-\beta x}\right)''=\beta^2e^{-\beta x}(\beta x-2),$$ efectivament tenim un màxim quan $\beta x=1$.
En el cas que ens ocupa, prenem $x=\frac{|f_2-f_1|}{\min(f_1,f_2)}$. És a dir, prenem $x$ com la resta normalitzada de les dues freqüències. Com que sabem que el màxim s'assoleix quan $\frac{|f_1-f_2|}{\text{CBW}(f_m)}=0.25$, tindrem que:
$$\frac{|f_1-f_2|}{\text{CBW}(f_m)}=0.25\iff\frac{\min(f_1, f_2)}{\text{CBW}(f_m)\cdot 0.25}\cdot\frac{|f_1-f_2|}{\min(f_1, f_2)}=1=\beta x\iff\beta=\beta(f_1,f_2)=\frac{\min(f_1, f_2)}{\text{CBW}(f_m)\cdot 0.25}$$
Així doncs, per a la funció de dissonància sabem que: $$\delta(s_1,s_2)=\delta((f_1,a_1),(f_2,a_2))\propto\beta(f_1,f_2)xe^{-\beta(f_1,f_2)x}$$ Recordant dues de les propietats necessàries (\ref{delta2} i \ref{delta3}) que ha de complir la funció $\delta$, deduïm que: $$\delta((f_1,a_1),(f_2,a_2))\propto a_1a_2$$ Finalment, la nostra funció de dissonància $\delta$ triada és: $$\delta((f_1,a_1),(f_2,a_2))=a_1a_2\beta(f_1,f_2)\frac{|f_2-f_1|}{\min(f_1,f_2)}e^{-\beta(f_1,f_2)\frac{|f_2-f_1|}{\min(f_1,f_2)}}$$
O, equivalentment, si simplifiquem l'expressió i l'escalem de tal forma que la màxima dissonància valgui 1 quan les amplituds valen 1, obtenim:
\begin{equation}
  \delta((f_1,a_1),(f_2,a_2))=a_1a_2\frac{|f_2-f_1|}{\text{CBW}(f_m)\cdot 0.25}e^{1-\frac{|f_2-f_1|}{\text{CBW}(f_m)\cdot 0.25}}
  \label{for:dissonancia}
\end{equation}
Observem que, pels comentaris fets prèviament en les seccions \ref{teoria_musical} i \ref{teoria_auditiva}, la funció $\delta$ no té unitats.\par\noindent A la figura \ref{main:fig3} es mostra una gràfica d'aquesta fórmula per a diferents sons purs:
\begin{center}
  \includestandalone[mode=image|tex,width=0.5\linewidth]{Imatges_main/model2}
  \captionof{figure}{Representació de la funció $\delta(s_1,s_2)$, on $s_1=(f_1,1)$ i $s_2=(f_2,1)$, a partir de diferents freqüències fixades $f_1$ i fent variar $f_2$ o, més ben dit, fent variar $\frac{f_2}{f_1}$ suposant $f_2\geq f_1$.}
  \label{main:fig3}
\end{center}
\subsection{Anàlisi dels resultats}
Amb la implementació de la funció $\delta$ en l'apartat anterior, ja podem treure les primeres conclusions. Per això, construïm ara un exemple d'aplicació d'aquest model.\par Suposem que estem treballant amb sons complexos que són notes musicals, és a dir, amb conjunts de sons simples les freqüències dels quals són múltiples enters d'una freqüència fonamental. Suposem, a més, que l'amplitud de l'harmònic $k$-èssim és $1/k^\alpha$, on $\alpha\in\mathbb{R}$ és un paràmetre fixat arbitrari. És a dir, la freqüència fonamental té amplitud 1, el segon harmònic té amplitud $1/2^\alpha$; el tercer, $1/3^\alpha$... Si fixem una nota musical $\N_1$ (de freqüència fonamental $f_1$) i fem variar-ne una altra $\N_2$ (de freqüència fonamental $f_2$) a partir d'un índex $r\in[1,2]$ tal que $f_2=rf_1$, obtenim els resultats que es mostren a la figura \ref{fig:complex}. Cal mencionar que en el nostre treball hem agafat $\alpha=0.75$. A més, les imatges de la figura \ref{fig:complex} han estat fetes considerant 9 harmònics de cada nota musical. Això últim ho hem fet a causa del ràpid decreixement dels harmònics a partir del desè, i la tria $\alpha$ ha estat feta de manera que l'amplitud dels 9 primers harmònics s'aproximés relativament bé. A la imatge següent es mostra el comportament d'aquests harmònics.
\begin{figure}[ht]
  \centering
  \includegraphics[width=0.5\linewidth]{Imatges_main/espectre_freq.png}
  \caption{Representació de l'espectre de freqüències de la nota LA4 (440 Hz) juntament amb la funció $\frac{1}{k^{0.75}}$ aproximant l'amplitud dels primers 9 harmònics}
\end{figure}\par
En els gràfics següents podem observar les \textit{corbes de dissonància} creades a partir del model descrit anteriorment:
\begin{center}
  \includestandalone[mode=image|tex,width=0.49\linewidth]{Imatges_main/complex1}
  \includestandalone[mode=image|tex,width=0.49\linewidth]{Imatges_main/complex2}\\
  \includestandalone[mode=image|tex,width=0.49\linewidth]{Imatges_main/complex3}
  \captionof{figure}{Dissonància entre dues notes musicals $\N_1$ i $\N_2$. Aquí hem fixat la freqüència fonamental de $\N_1$ a 110 Hz, 440 Hz i 1760 Hz, respectivament i hem fet variar la relació $f_2/f_1$, essent $f_2$ la freqüència fonamental de $\N_2$.}
  \label{fig:complex}
\end{center}\par
Observem la peculiar forma dels gràfics. En particular, sembla que se satisfà que els punts de màxima consonància es troben quan el quocient de les dues freqüències fonamentals de les notes $\N_1$ i $\N_2$ és una fracció involucrant nombres enters senzills, tal com ja teoritzaven el pitagòrics al segle \MakeUppercase{\romannumeral 6} aC. Això es deu al següent fet: considerem dues notes musicals $\N_1,\N_2\in\mathcal{N}$ amb freqüències fonamentals $f_1$ i $f_2$, respectivament, tals que $\frac{f_2}{f_1}=\frac{p}{q}$, on $p,q\in\NN$ són nombres enters positius i coprimers. És a dir, $pf_1=qf_2$. Això significa que la freqüència de l'harmònic $p$-èssim de $f_1$ serà la mateixa que la de l'harmònic $q$-èssim de $f_2$, i com a conseqüència $\delta((pf_1,a_1),(qf_2,a_2))=0$ independentment dels valors $a_1$ i $a_2$, per la definició de la funció $\delta$. De fet, això últim ocorrerà sempre que es comparin dos sons simples de freqüències $kpf_1$ i $kqf_2$, $k\in\NN$. Això comporta que si els valors de $p$ i $q$ són petits, aquestes igualtats entre harmònics s'aniran repetint constantment, el que provocarà un no-creixement en el valor total de la dissonància de $\N_1\oplus\N_2$. En canvi, si $p$ i $q$ són valors grans, la igualtat entre els harmònics ocorrerà quan n'haguem considerat molts i, per tant, a cada sumand de l'equació \eqref{eq_final} sempre anirem aportant a la dissonància total petites, però no nu\lgem es, contribucions, que al final provocaran un augment substancial del valor de $D(\N_1\oplus\N_2)$.\par
D'altra banda, notem que a mesura que augmentem la freqüència base $f_1$, la dissonància en general va disminuint. Això és degut a la forma de la funció $\delta$ i també al comportament de la funció $\text{CBW}$ a freqüències altes. Aquest fet explica per què no se sol produir música amb notes musicals que tenen freqüències fonamentals menors a 100 Hz.\par Feta aquesta anàlisi i havent vist que en general la funció dissonància perd magnitud a mesura que considerem sons més aguts, si $\N_1$ i $\N_2$ són dos sons amb freqüències fonamentals $f_1$ i $f_2$ respectivament, ens podem preguntar què ocorre quan fixem una fracció $\frac{p}{q}$, on $p,q\in\NN$ i considerem que el quocient de freqüències $\frac{f_2}{f_1}$ és igual a $\frac{p}{q}$. És a dir, quina dissonància percebem quan fixem $\frac{f_2}{f_1}=\frac{p}{q}$. És clar que la dissonància $D(\N_1\oplus\N_2)$ resultant d'haver combinat aquests dos sons amb la relació de freqüències mencionada variarà de forma decreixent conforme augmentem el valor de $f_1$. Per tal de poder percebre amb més claredat com és aquest decreixement, al gràfic de la figura \ref{fraccio_pq} l'hem representat a partir de quatre fraccions $\frac{p}{q}$ i fent variar $f_1$ en l'interval $[60,1000]$:
\begin{figure}[ht]
  \centering
  \includestandalone[mode=image|tex,width=0.6\linewidth]{Imatges_main/grafic_fraccio_pq}
  \caption{Variació de la dissonància de les notes musicals $\N_1$ i $\N_2$ amb respectives freqüències fonamentals $f_1$ i $f_2$, imposant que $\frac{f_2}{f_1}=\frac{p}{q}$, per a diferents fraccions $\frac{p}{q}$.}
  \label{fraccio_pq}
\end{figure}
\section{Anàlisi dels fets empírics}
Per corroborar que el model era encertat, vam dur a terme un test a una mostra d'aproximadament 200 persones. El test consistia a escoltar 11 sons diferents, 3 corresponents a combinacions de freqüències baixes $(f \approx 110\Hz)$, 4 corresponents a freqüències mitjanes $(f \approx 440\Hz)$ i 4 corresponents a freqüències altes $(f \approx 1760\Hz)$. L'oient havia de qualificar-los amb un nombre de l'1 al 10, essent 1 el nivell més dissonant i 10 el menys dissonant. El test també demanava a l'enquestat l'edat i, més important encara, la seva formació musical, que podia variar entre 5 nivells diferents (molt baix, baix, intermedi, alt i molt alt). Per tal de verificar que la gent era honesta amb la seva puntuació sobre el nivell de formació musical vam demanar que descrivissin breument la seva trajectòria musical. \par
En total vam aconseguir 190 respostes, de les quals 42 eren de nivell molt baix; 69, de nivell baix; 43, de nivell intermedi; 26, de nivell alt, i 10, de nivell molt alt. Aquesta separació entre grups ens va permetre poder classificar les dades en 3 nivells: nivell molt baix/baix (111 respostes), nivell intermedi (43 respostes) i nivell alt/molt alt (36 respostes). \par
Al test únicament vam incloure combinacions de dues notes $\N_1$ i $\N_2$ de l'escala musical d'un piano. Per a la primera nota vam escollir sempre la nota LA de freqüències 110 Hz, 440 Hz i 1760 Hz, respectivament segons el rang de freqüències (baixes, mitjanes o altes) que estiguéssim treballant.\par A continuació, veiem la freqüència (i la seva nota corresponent) amb què vam combinar aquestes notes i quin hauria de ser el resultat obtingut segons el nostre model:
\begin{table}[ht]
  \centering
  \begin{tabular}{| c | c | c | c |}
    \cline{2-4}
    \multicolumn{1}{c|}{} & $\N_1$                               & $\N_2$               & $D(\N_1\oplus\N_2)$ \\
    \hline
    \hline
    So 1                  & \multirow{3}{2.5cm}{LA2 $(110\Hz)$}  & SI2 $(123.47\Hz)$    & 2.1409              \\
    \cline{1-1}\cline{3-4}
    So 2                  &                                      & MI3 $(164.81\Hz)$    & 0.7900              \\
    \cline{1-1}\cline{3-4}
    So 3                  &                                      & SOL3 $(196\Hz)$      & 1.2839              \\
    \hline
    So 4                  & \multirow{4}{2.5cm}{LA4 $(440\Hz)$}  & LA\#4 $(466.16\Hz)$  & 1.8835              \\
    \cline{1-1}\cline{3-4}
    So 5                  &                                      & RE5 $(587.33\Hz)$    & 0.3891              \\
    \cline{1-1}\cline{3-4}
    So 6                  &                                      & RE\#5 $(622.25\Hz)$  & 0.6032              \\
    \cline{1-1}\cline{3-4}
    So 7                  &                                      & FA5 $(698.46\Hz)$    & 0.5752              \\
    \hline
    So 8                  & \multirow{4}{2.5cm}{LA6 $(1760\Hz)$} & LA\#6 $(1864.66\Hz)$ & 1.2451              \\
    \cline{1-1}\cline{3-4}
    So 9                  &                                      & DO\#7 $(2217.46\Hz)$ & 0.2617              \\
    \cline{1-1}\cline{3-4}
    So 10                 &                                      & MI7 $(2637.02\Hz)$   & 0.09559             \\
    \cline{1-1}\cline{3-4}
    So 11                 &                                      & FA\#7 $(2959.96\Hz)$ & 0.2156              \\
    \hline
  \end{tabular}
  \caption{Freqüències usades al test i dissonància predita pel nostre model}
  \label{tab:3}
\end{table}\par
\noindent Els gràfics següents mostren de manera més clara on se situen les combinacions de notes escollides:
\begin{center}
  \includestandalone[mode=image|tex,width=0.49\linewidth]{Imatges_main/complex1_marcat}
  \includestandalone[mode=image|tex,width=0.49\linewidth]{Imatges_main/complex2_marcat}\\
  \includestandalone[mode=image|tex,width=0.49\linewidth]{Imatges_main/complex3_marcat}
  \captionof{figure}{Gràfics on es mostra la posició de les combinacions de sons escollides pel test en les corbes de dissonància}
\end{center}
\noindent El següent gràfic mostra els resultats obtinguts en el test:\par
\begin{figure}[ht]
  \centering
  \includestandalone[mode=image|tex,width=0.53\linewidth]{Imatges_main/mediana}
  \caption{Resultats del test juntament amb la predicció (reescalada convenientment) dels nostres resultats}
\end{figure}
Per a l'anàlisi dels resultats ens hem basat en la mediana dels tots els valors obtinguts. El motiu de no haver inclòs la mitjana o la moda juntament amb la mediana es deu al fet que la mitjana està molt condicionada pels valors centrals, de manera que les puntuacions mitjanes de cada so eren totes relativament similars, i no haurien de ser-ho per l'elecció feta dels sons. Pel que fa a la moda, aquesta ens va condicionar molt els valors extrems, ja que molta gent puntuava la dissonància dels sons amb valors molt extrems: o propers a 1, o propers a 10.\par
Observem que hi ha clares dissemblances entre els diferents nivells. En particular, deduïm que quan més familiar ens és un so, millor són els resultats predits pel model. Efectivament, en els sons 4, 5, 6 i 7, que són els de freqüències mitjanes (i.e. els més comuns), l'error comès en l'aproximació teòrica és inferior en comparació amb l'aproximació del model dels sons greus (sons 1, 2 i 3) i els sons aguts (sons 8, 9, 10 i 11). Podem concloure, doncs, que els resultats empírics observats verifiquen prou bé els resultats obtinguts al nostre model.\par Finalment, cal comentar també que, tot i haver-ho considerat, al final no vam fer cap distinció per rangs d'edat. Ho vam decidir així perquè no vam posar sons extremadament greus ni extremadament aguts, que són els que es deixen de sentir abans. Això juntament amb el fet que hi va haver molt pocs enquestats de més de 65 anys (que és quan l'oïda comença a deteriorar-se substancialment \cite{age_hearing}), en comparació amb el resta de rangs d'edat, ens va fer descartar aquesta possibilitat.
\section{Conclusions i possibles refinaments}
Malgrat que la dissonància és un tema força subjectiu, al llarg del treball hem vist que és possible modelar la dissonància partint de fenòmens auditius com és el de la banda crítica.\par
Una possible extensió és endinsar-nos més en l'estructura algebraica que hi ha darrere el conjunt de sons complexos. Per això caldria, però, redefinir el conjunt de sons simples i incorporar-hi un nou factor: la fase d'un so. Això ens permetria, juntament amb petites modificacions, definir les funcions $d$ i $D$ sobre un espai vectorial de dimensió infinita\footnote{De fet, el conjunt no tan sols seria infinit, sinó també no-numerable.}. D'aquesta manera, $d$ seria una forma bilineal i $D$ la seva forma quadràtica associada, ambdues definides sobre l'espai vectorial en qüestió.\par
Referent al test, a causa de la pandèmia no vam poder dur-lo a terme amb les condicions idònies que hauríem volgut. La forma ideal de fer aquest test hauria estat assegurar-nos que tothom escoltés els sons amb els mateixos auriculars i en un ambient silenciós, per tal d'eliminar possibles factors externs que interferissin en els resultats.\par
També per millorar l'efectivitat del test, podríem haver experimentat amb combinacions de tres o quatre notes musicals tocades simultàniament, que de fet són les que solem sentir quan escoltem una melodia d'un piano, per exemple. \par Para\lgem elament a aquest darrer tema, també podríem haver considerat la possibilitat d'incloure diversos instruments i fer un estudi de com varia la percepció de la dissonància segons la classe d'instrument (de vent, de corda o de percussió) que estiguem considerant, amb l'objectiu de descobrir quin d'ells és aproximat pel nostre model més satisfactòriament.
\section{Agraïments}
Primer de tot agraïm a la Natàlia Castellana Vila, tutora del treball, per la seva exigència i el constant suport al llarg de tot el projecte. \par
D'altra banda, agraïm també al professor de l'assignatura, Xavier Mora Giné, per la increïble implicació en aquest treball, així com la participació en el debat sobre les qüestions formals que ha implicat aquest treball.\par
A més, i no menys important, agraïm a tots aquells professors de l'assignatura que han aportat i compartit les seves crítiques i idees amb el propòsit de millorar i consolidar el treball.\par
Finalment, volem deixar constància del nostre agraïment per a totes aquelles persones que van dur a terme el test per tal d'ajudar-nos en el treball.
\printbibliography[heading=bibintoc]
\appendix
\section{Demostracions}\label{demos}
\begin{prop*}[\ref*{prop_dem1}]
  Siguin $\X,\Y,\Z\in\mathcal{C}$ sons complexos i $\lambda\in[0,\infty)$. Llavors:
  \begin{enumerate}[label=$d$\arabic*),ref=$d$\arabic*]
    \item\label{dd1} $d(\X,\Y)=d(\Y,\X)$.
    \item\label{dd2} $d(\X\oplus\Y,\Z)=d(\X,\Z)+d(\Y,\Z)$.
    \item\label{dd3} $d(\X,\Y\oplus\Z)=d(\X,\Y)+d(\X,\Z)$.
    \item\label{dd4} $d(\lambda\cdot\X,\Y)=\lambda d(\X,\Y)$.
    \item\label{dd5} $d(\X,\lambda\cdot\Y)=\lambda d(\X,\Y)$.
  \end{enumerate}
\end{prop*}
\begin{proof}
  Suposem que $\X=\{x_i\}_{i=1}^n$, $\Y=\{y_i\}_{i=1}^m$ i $\Z=\{z_i\}_{i=1}^p$.
  \begin{enumerate}[label=$d$\arabic*)]
    \item $$d(\X,\Y)=\frac{1}{2}\sum_{i=1}^n\sum_{j=1}^m\delta(x_i,y_j)=\frac{1}{2}\sum_{j=1}^m\sum_{i=1}^n\delta(y_j,x_i)=d(\Y,\X),$$ on hem utilitzat que la funció $\delta$ és simètrica (propietat \ref{delta1}).
    \item Per simplificar la notació, anomenem
          $$
            w_i=\left\{
            \begin{array}{ccc}
              x_i     & \text{si} & 1\leq i\leq n   \\
              y_{i-n} & \text{si} & n+1\leq i\leq m
            \end{array}\right.
          $$ Aleshores tenim que: $$d(\X\oplus\Y,\Z)=\frac{1}{2}\sum_{i=1}^{n+m}\sum_{j=1}^p\delta(w_i,z_j)=\frac{1}{2}\sum_{i=1}^n\sum_{j=1}^p\delta(x_i,z_j)+\frac{1}{2}\sum_{i=n+1}^m\sum_{j=1}^p\delta(y_{i-n},z_j)=d(\X,\Z)+d(\Y,\Z).$$
    \item És conseqüència de les propietats \ref{dd1} i \ref{dd2}.
    \item Tenim que: $$d(\lambda\cdot\X,\Y)=\frac{1}{2}\sum_{i=1}^n\sum_{j=1}^m\delta(\lambda\cdot x_i,y_j)=\frac{\lambda}{2}\sum_{i=1}^n\sum_{j=1}^m\delta(x_i,y_j)=\lambda d(\X,\Y),$$ on hem aplicat la propietat \ref{delta2}.
    \item És conseqüència de les propietats \ref{dd1} i \ref{dd4}.
  \end{enumerate}
\end{proof}
\begin{prop*}[\ref*{prop_dem2}]
  Siguin $\X,\Y\in\mathcal{C}$ sons complexos i $\lambda\in[0,\infty)$. Llavors:
  \begin{enumerate}[label=$D$\arabic*),ref=$D$\arabic*]
    \item\label{DD1} $D(\X\oplus\Y)=D(\X)+D(\Y)+2d(\X,\Y)$.
    \item\label{DD2} $D(\lambda\cdot \X)=\lambda^2D(\X)$.
  \end{enumerate}
\end{prop*}
\begin{proof}
  \hfill
  \begin{enumerate}[label=$D$\arabic*)]
    \item Aplicant les propietats \ref{dd1}, \ref{dd2} i \ref{dd3}, obtenim:
          \begin{multline*}
            D(\X\oplus\Y)=d(\X\oplus\Y,\X\oplus\Y)=d(\X,\X\oplus\Y)+d(\Y,\X\oplus\Y)=d(\X,\X)+d(\X,\Y)+\\+d(\Y,\X)+d(\Y,\Y)=d(\X,\X)+d(\Y,\Y)+2d(\X,\Y)=D(\X)+D(\Y)+2d(\X,\Y).
          \end{multline*}
    \item Aplicant les propietats \ref{dd4} i \ref{dd5}, obtenim: $$D(\lambda\cdot\X)=d(\lambda\cdot\X,\lambda\cdot\X)=\lambda d(\X,\lambda\cdot\X)=\lambda^2d(\X,\X)=\lambda^2D(\X).$$
  \end{enumerate}
\end{proof}
\begin{corollary*}[\ref*{coro_dem3}]
  Siguin $\X_1,\ldots,\X_n\in\mathcal{C}$ sons complexos i $\lambda\in[0,\infty)$. Aleshores: $$D(\X_1\oplus\cdots\oplus\X_n)=\sum_{i=1}^nD(\X_i)+\sum_{\substack{i,j=1\\i\ne j}}^nd(\X_i,\X_j)=\sum_{i,j=1}^nd(\X_i,\X_j).$$
\end{corollary*}
\begin{proof}
  Ho demostrarem per inducció sobre $n$. Per $n=1$ clarament es compleix la igualtat.\par
  \noindent Suposem certa la igualtat per $n=k$ i demostrem-la per $n=k+1$. Per les propietats \ref{DD1} i \ref{dd2} tenim que:
  \begin{multline*}
    D(\X_1\oplus\cdots\oplus\X_{k+1})=D([\X_1\oplus\cdots\oplus\X_k]\oplus\X_{k+1})=D(\X_1\oplus\cdots\oplus\X_k)+D(\X_{k+1})+\\+2d(\X_1\oplus\cdots\oplus\X_k,\X_{k+1})=D(\X_1\oplus\cdots\oplus\X_k)+D(\X_{k+1})+2\sum_{i=1}^kd(\X_i,\X_{k+1}).
  \end{multline*}
  Ara bé, per la hipòtesi d'inducció tenim que:
  $$D(\X_1\oplus\cdots\oplus\X_k)=\sum_{i=1}^kD(\X_i)+\sum_{\substack{i,j=1\\i\ne j}}^kd(\X_i,\X_j).$$
  Així doncs, ajuntant-ho tot, obtenim:
  \begin{multline*}
    D(\X_1\oplus\cdots\oplus\X_{k+1})=\sum_{i=1}^kD(\X_i)+\sum_{\substack{i,j=1\\i\ne j}}^kd(\X_i,\X_j)+D(\X_{k+1})+2\sum_{i=1}^kd(\X_i,\X_{k+1})=\\=\sum_{i=1}^{k+1}D(\X_i)+\sum_{\substack{i,j=1\\i\ne j}}^{k+1}d(\X_i,\X_j).
  \end{multline*}
  Finalment, si en aquesta igualtat expressem $D(\X_i)$ com $d(\X_i,\X_i)$, deduïm la segona igualtat del coro\lgem ari.
\end{proof}
\end{document}
